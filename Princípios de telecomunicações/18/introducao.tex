\newpage
\section{Introdução}
Este trabalho tem como objetivo analisar o chaveamento por deslocamento de fase em quadratura (QPSK) em sistemas de comunicação digital. Diferente do PSK com duas fases, a técnica QPSK usa quatro fases de um sinal de portadora para transmitir os dados.  Isso traz uma vantagem em relação ao PSK, pois a banda necessária, teoricamente, é reduzida pela metade. Porém, a taxa de transmissão de dados não é duplicada, visto que é necessário uma maior precisão nas medidas de fase da portadora. Assim, é necessário diminuir a frequência de clock para melhorar a relação sinal ruído, diminuindo a taxa de erro de bits.

