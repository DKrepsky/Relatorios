\newpage

\section{Introdução}
Ao se transportar um sinal de uma fonte para uma carga, caso as mesmas não possuam a mesma impedância, ocorre a reflexão de parte do sinal enviado de volta para a fonte \cite{Rhea}. Este fenômeno é bastante prejudicial em sistemas de comunicação, pois reduz a potência transmitida e, em alguns casos, pode ocasionar danos a fonte. Para solucionar este problema são utilizadas os casadores de impedância, também chamados de rede adaptadora de impedância. Seu princípio de funcionamento consiste em utilizar elementos passivos de modo a satisfazer as condições do teorema da máxima transferência de potência. 

As topologias utilizadas são as redes do tipo $L$, $T$ e $\pi$, sendo possível o acoplamento de redes $L$ em cascata para melhorar o desempenho em sinais de banda larga.
O objetivo deste trabalho é analisar o projeto e desempenho de redes do tipo $L$ e $T$, sendo também estudado o comportamento de redes $L$ em cascata.