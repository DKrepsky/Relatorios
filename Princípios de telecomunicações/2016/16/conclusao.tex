\newpage
\section{Discussão e Conclusão}

	Neste trabalho foi analisado o processo de modulação de um sinal de dados com a técnica de chaveamento por deslocamento defase (PSK) com codificação bi-fase, bem como a recuperação da informação utilizando um demodulador PSK.
	
	Foi constatado que a topologia utilizada para modulação e demodulação é capaz de recuperar os dados com uma baixa BER caso o deslocamento de fase seja menor que $\pm 90º$. Entretanto, caso o deslocamento de fase seja maior que $\pm 90º$, o circuito demodulador não consegue recuperar a informação, apresentando na saída o inverso dos dados enviados.
	
\newpage
\section{Referências}
[1] Jacob, J. L., Roteiro de laboratório, Universidade estadual de londrina, 2017. Disponível em: http://www.uel.br/pessoal/jaimejacob/pages/arquivos/LabPrincCom/Lab16\_PSK.pdf. Acesso em 15/02/2017.

[2] Abrão, T., Modulação e Sistemas de Comunicação Digital, Universidade estadual de londrina, 2017. Não publicado.