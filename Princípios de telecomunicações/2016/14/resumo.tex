\newpage
\begin{abstract}
\addcontentsline{toc}{section}{Resumo}

Neste trabalho foi analisado a técnica de regeneração de clock através do uso da codificação bi-fase. A metodologia adotada consiste em gerar uma palavra de dados e utilizar um clock de 80kHz para negar o sinal em meios ciclos alternados. A partir do sinal bi-fase, um derivador é utilizado para gerar pulsos na metade de cada período de bit. Esses pulsos são utilizados para reconstrução do clock de bit e, posteriormente, a recuperação dos dados em formato NRZ. Para limpar o sinal NRZ resultante, a técnica \textit{integrate and dump} é aplicada. Com a execução desse trabalho, foi possível observar o envio de dados sem a necessidade de uma linha extra para o sinal de clock. Contudo, os dados recuperados possuem 1 bit de atraso na recepção.
\end{abstract}
