\documentclass[12pt,a4paper]{article}%
%Options -- Point size:  10pt (default), 11pt, 12pt
%        -- Paper size:  letterpaper (default), a4paper, a5paper, b5paper
%                        legalpaper, executivepaper
%        -- Orientation  (portrait is the default)
%                        landscape
%        -- Print size:  oneside (default), twoside
%        -- Quality      final(default), draft
%        -- Title page   notitlepage, titlepage(default)
%        -- Columns      onecolumn(default), twocolumn
%        -- Equation numbering (equation numbers on the right is the default)
%                        leqno
%        -- Displayed equations (centered is the default)
%                        fleqn (equations start at the same distance from the right side)
%        -- Open bibliography style (closed is the default)
%                        openbib
% For instance the command
%           \documentclass[a4paper,12pt,leqno]{article}
% ensures that the paper size is a4, the fonts are typeset at the size 12p
% and the equation numbers are on the left side
%====================Tabelas==============================================
\usepackage{multirow}
%=============================Símbolos Matemáticos=====================================================================
\usepackage{amsmath}
\usepackage{amsfonts}
\usepackage{amssymb}
%=======================================Figuras===========================================================
\usepackage{graphicx}
%\usepackage{wrapfig}
\usepackage{float}
%=====================================Língua e acentos=============================================================
\usepackage[brazil]{babel}
\usepackage[utf8]{inputenc}
\usepackage[T1]{fontenc}
%========================================Espaçamento==========================================================
\usepackage[top=3cm, bottom=2cm, left=2cm, right=2cm]{geometry}
\usepackage{indentfirst}
%=======================================Lista de códigos===========================================================
\usepackage{listings}                   % para formatar código-fonte
\lstset{numbers=left, numberstyle=\tiny, stepnumber=1, numbersep=5pt, basicstyle=\scriptsize , frame=trbl}
%======================================Latexdraw============================================================
%\usepackage[usenames,dvipsnames]{pstricks}
%\usepackage{epsfig}
%\usepackage{pst-grad} % For gradients
%\usepackage{pst-plot} % For axes
%==================================================================================================
%-------------------------------------------

\usepackage[numbers]{natbib}
\usepackage[numbib]{tocbibind}

\begin{document}

\begin{titlepage}
\begin{center}
\begin{figure}[h]
\includegraphics[scale=0.76]{imagem/topdotitulo.png}
\end{figure}
\rule{\columnwidth}{1.5mm}
\

\large Italo Jackson de Souza Gloor\\
\large Lucas Felipe de Lima \\
\large David Maykon Krepsky Silva

\vspace{4cm}
{\bf \Large Experiência 11 - Modem FM}
\vspace{3.5cm}

\begin{flushright}
Data de realização do experimento:\\
29 de Outubro de 2015\\
Série/Turma:\\
1000/1011\\
Prof. Dr. Jaime Laelson Jacob 
\end{flushright}

\vspace{3.2cm}
\today

\rule{\columnwidth}{1.3mm}
\end{center}
\end{titlepage}
\newpage

\begin{abstract}
\addcontentsline{toc}{section}{Resumo}

Neste trabalho foi realizado o estudo de filtros passivos compostos indutores e capacitores (filtros LC) de forma a analisar a resposta em frequência para filtro passa baixas (FPB) e filtro passa alta (FPA). A metodologia utilizada consiste em realizar o projeto do filtro normalizado, transformar para o tipo de filtro requerido e simular o circuito no software Orcad.
Durante o laboratório foi possível observar as diferentes respostas em frequência para cada tipo de filtro estudado.Também foi visualizado a melhora no fator Q de filtros em cascata em relação a um único filtro de ordem mais alta.
\end{abstract}
\input{3_sumario.tex}
\newpage

\section{Introdução}
O experimento tem como objetivo desenvolver o conhecimento dos alunos sobre filtros passivos, em específico, filtros do tipo LC. Tais filtros são fundamentas para a área de engenharia elétrica sendo amplamente utilizados nos campos de telecomunicações, instrumentação, controle e etc. 


\newpage

\section{Teoria}
\subsection{Filtros Passivos}
Filtros passivos são circuitos que removem uma porção indesejada do sinal sem inserir energia no mesmo. São compostos por resistores, capacitores e indutores que utilizam as propriedades de armazenamento de energia (em forma de campo elétrico nos capacitores e campo magnético nos indutores) para alterar a amplitude do sinal de acordo com a frequência. Os filtros ativos diferem dos passivos pois possuem eletrônica de modo a amplificar (aumentar a energia) do sinal, porém, para frequências muito altas o uso de filtros ativos se torna inviável, dado a grande quantidade de capacitância parasita nos dispositivos semi-condutores.

Os filtros passivos são classificados de acordo com a faixa de frequências a qual o filtro atenua, sendo elas:
\begin{itemize}
    \item \textbf{Passa-Baixas (FPB)} o qual permite a passagem das frequências abaixo de $f_c$ (frequência de corte);
    
    \item \textbf{Passa-Altas (FPA)} o qual permite a passagem das frequências acima de $f_c$;
    
    \item \textbf{Passa-Faixa (FPF)} que atenua frequências abaixo de $f_1$ e frequências acima de $f_2$;
    
    \item \textbf{Rejeita-Faixa (FRF)} que permite a passagem de frequencias entre $f_1$ e $f_2$.
    
\end{itemize}

A figura \ref{fig:tipos} mostra os tipos de resposta em frequência para os 
filtros citados.

\begin{figure}[!h]
  \centering
  \caption{Resposta em frequência para filtros FPA, FPB, FPF e FRF}
  \includegraphics[scale=0.6]{Imagens/types.png}
  
  \label{fig:tipos}
  \small Fonte: www.dreamitdesignitbuildit.wordpress.com.
\end{figure}

Uma segunda classificação para os filtros é relacionada ao \textit{ripple} e a defasagem da resposta em frequência. Os tipos mais comuns empregados na prática são:

\begin{itemize}
    \item \textbf{Butterworth};
    \item \textbf{Chebyshev (tipo I ou II)};
    \item \textbf{Bessel}.
\end{itemize}

A figura \ref{fig:tipos2} mostra as características da resposta em frequência 
para os filtros citados acima.

\begin{figure}
  \centering
  \caption{Características dos filtros Butterworth, Chebyshev, Bessel e 
  Elíptico}
  \includegraphics[scale=0.6]{Imagens/types2.jpg}
  
  \label{fig:tipos2}
  \small Fonte: http://www.circuitstoday.com
\end{figure}

\subsection{Frequência de Corte e Largura de Banda}
A frequência de corte ($f_c$ é a frequência para qual o filtro apresentará uma atenuação de 3dB e é o parâmetro fundamental para o projeto de filtros.
Outro parâmetro importante para os filtros do tipo passa-faixa e rejeita-faixa é a largura da banda de passagem, a qual é composta por uma frequência de corte inferior (denominada $f_1$) e uma superior ($f_2$).

\newpage

\section{Metodologia Experimental}

\subsection{FPB}

Projetar um FPB com resposta Butterworth de 3a. ordem utilizando apenas um 
indutor, terminações como no caso anterior, isto é, $R_S = 50 \Omega$ e $R_L = 
470 \Omega$ e $f_c = 5,4 \ kHz$. Implemente e caracterize a resposta em 
frequência do filtro passivo, determinando experimentalmente os parâmetros que 
caracterizam o filtro FPB:

\begin{itemize}
  \item frequência de corte;
  \item atenuação fora da faixa de passagem (dB/década);
  \item atenuação na faixa de passagem;
  \item defasagem ao longo de toda a faixa de frequências (de passagem e 
  rejeição).
\end{itemize}

\subsection{FPA}

Refaça o item anterior para um FPA com resposta Butterworth de
3a. ordem utilizando apenas um indutor, terminações $R_S = 470\Omega$ e $R_L = 
50\Omega$ e $f_c = 4,6 \ kHz$. Implemente e caracterize a resposta em 
frequência do filtro passivo, anotando os parâmetros que caracterizam o filtro 
FPA.

\subsection{FPF Cascata}
Conecte os dois filtros em série (cascata) observando as impedâncias e meça a 
resposta em frequência do conjunto. Qual a função de transferência 
correspondente? Quais a(s) nova(s) frequência(s) de corte.

\subsection{FPF}
Projete um filtro Butterworth com a função de transferência resultante da 
associação dos filtros do item anterior a partir dos valores tabelados para os
elementos LC de protótipo.

\begin{itemize}
  \item implemente novamente o filtro, agora utilizando os elementos de projeto 
  do item 4.
  \item compare a resposta em frequência (módulo) com a obtida no item 3a.
\end{itemize}

\newpage

\section{Resultados e Análise de Dados}

\subsection{FPB}
O primeiro filtro projetado foi um filtro passa-baixas de terceira ordem, com 
resposta do tipo Butterworth utilizando apenas 1 indutor.

A figura \ref{fig:fpb-norm} mostra o circuito normalizado.
\begin{figure}[!h]
  \centering
  
  \includegraphics[scale=0.4]{Imagens/fpb-norm}
  \label{fig:fpb-norm}
  \caption{Filtro passa-baixas normalizado.}
\end{figure}


A figura \ref{fig:fpb} mostra o circuito já desnormalizado, pronto para 
simulação.
\begin{figure}[!h]
  \centering
  
  \includegraphics[scale=0.4]{Imagens/fpb}
  \label{fig:fpb}
  \caption{Filtro desnormalizado passa-baixas.}
\end{figure}

A resposta em frequência obtida está na figura \ref{fig:resp_freq}, onde foi 
obtido uma frequência de corte de 5,317 kHz. 

\begin{figure}[!h]
  \centering
  
  \includegraphics[scale=0.3]{Imagens/resp_freq}
  \label{fig:resp_freq}
  \caption{Resposta em frequência do filtro passa-baixas.}
\end{figure}

A figura \ref{fig:resp_freq_db} mostra a resposta em dB, nota-se que a 
atenuação aumenta em aproximadamente 60 dB por década, o que corresponde a 
ordem 3 do filtro.

\begin{figure}[!h]
  \centering
  
  \includegraphics[scale=0.3]{Imagens/resp_freq_db}
  \label{fig:resp_freq_db}
  \caption{Resposta em frequência do filtro passa-baixas em dB.}
\end{figure}

A figura \ref{fig:resp_freq_phase} mostra a fase da resposta em frequência para 
o filtro passa baixas, onde é possível observar uma defasagem de 270 graus, o 
que condiz com a teoria pois o filtro é de ordem 3.

\begin{figure}[!h]
  \centering
  
  \includegraphics[scale=0.3]{Imagens/resp_freq_phase}
  \label{fig:resp_freq_phase}
  \caption{Fase da resposta em frequência do filtro passa-baixas.}
\end{figure}

\subsection{FPA}
O segundo filtro projetado foi um filtro passa-altas de terceira ordem, com 
resposta do tipo Butterworth utilizando apenas 1 indutor.

A figura \ref{fig:fpa-norm} mostra o circuito normalizado.
\begin{figure}[!h]
  \centering
  
  \includegraphics[scale=0.4]{Imagens/fpa-norm}
  \label{fig:fpa-norm}
  \caption{Filtro passa-altas normalizado.}
\end{figure}

A figura \ref{fig:fpa} mostra o circuito já desnormalizado, pronto para 
simulação.

\begin{figure}[!h]
  \centering
  
  \includegraphics[scale=0.4]{Imagens/fpa}
  \label{fig:fpa}
  \caption{Filtro desnormalizado passa-altas.}
\end{figure}

A resposta em frequência obtida está na figura \ref{fig:resp_freq_2}, onde foi 
obtido uma frequência de corte de 4,384 kHz. 

\begin{figure}[!h]
  \centering
  
  \includegraphics[scale=0.3]{Imagens/resp_freq_2}
  \label{fig:resp_freq_2}
  \caption{Resposta em frequência do filtro passa-altas.}
\end{figure}

A figura \ref{fig:resp_freq_db_2} mostra a resposta em dB, nota-se que a 
atenuação aumenta em aproximadamente 60 dB por década, o que corresponde a 
ordem 3 do filtro.

\begin{figure}[!h]
  \centering
  
  \includegraphics[scale=0.3]{Imagens/resp_freq_db_2}
  \label{fig:resp_freq_db_2}
  \caption{Resposta em frequência do filtro passa-altas em dB.}
\end{figure}

A figura \ref{fig:resp_freq_phase_2} mostra a fase da resposta em frequência 
para o filtro passa altas, onde é possível observar uma defasagem de 270 graus, 
o que condiz com a teoria pois o filtro é de ordem 3.

\begin{figure}[!h]
  \centering
  
  \includegraphics[scale=0.3]{Imagens/resp_freq_phase_2}
  \label{fig:resp_freq_phase_2}
  \caption{Fase da resposta em frequência do filtro passa-altas.}
\end{figure}

\subsection{FPF em cascata}
O terceiro filtro projetado foi um filtro passa-faixas em cascata de terceira 
ordem, com resposta do tipo Butterworth utilizando apenas 1 indutor.

A figura \ref{fig:fpf-cascata} mostra o circuito já desnormalizado, pronto para 
simulação.

\begin{figure}[!h]
  \centering
  
  \includegraphics[scale=0.4]{Imagens/fpf-cascata}
  \label{fig:fpf-cascata}
  \caption{Filtro desnormalizado passa-faixa em cascata.}
\end{figure}

A resposta em frequência obtida está na figura \ref{fig:resp_freq_3}, onde foi 
obtido uma frequência central de 4,655 kHz. 

\begin{figure}[!h]
  \centering
  
  \includegraphics[scale=0.3]{Imagens/resp_freq_3}
  \label{fig:resp_freq_3}
  \caption{Resposta em frequência do filtro passa-faixa em cascata.}
\end{figure}

A figura \ref{fig:resp_freq_db_3} mostra a resposta em dB, nota-se que a 
atenuação aumenta em aproximadamente 60 dB por década, o que corresponde a 
ordem 3 do filtro.

\begin{figure}[!h]
  \centering
  
  \includegraphics[scale=0.3]{Imagens/resp_freq_db_2}
  \label{fig:resp_freq_db_3}
  \caption{Resposta em frequência do filtro passa-faixa em cascata em dB.}
\end{figure}

A figura \ref{fig:resp_freq_phase_3} mostra a fase da resposta em frequência 
para o filtro passa altas, onde é possível observar uma defasagem de 270 graus, 
o que condiz com a teoria pois o filtro é de ordem 3.

\begin{figure}[!h]
  \centering
  
  \includegraphics[scale=0.3]{Imagens/resp_freq_phase_3}
  \label{fig:resp_freq_phase_3}
  \caption{Fase da resposta em frequência do filtro passa-faixa em cascata.}
\end{figure}

\subsection{FPF}
O quarto e ultimo filtro foi projetado para ter uma resposta em frequência 
semelhante a do filtro FPF. O objetivo foi de analisar as diferenças 
em se utilizar um filtro FPF e um filtro FPB em conjunto com um filtro FPA.

O terceiro filtro projetado foi um filtro passa-faixas em cascata de terceira 
ordem, com resposta do tipo Butterworth utilizando apenas 1 indutor.

A figura \ref{fig:fpf} mostra o circuito já desnormalizado, pronto para 
simulação.

\begin{figure}[!h]
  \centering
  
  \includegraphics[scale=0.4]{Imagens/fpf}
  \label{fig:fpf}
  \caption{Filtro desnormalizado passa-faixa.}
\end{figure}

A resposta em frequência obtida está na figura \ref{fig:resp_freq_4}, onde foi 
obtido uma frequência central de 4,655 kHz. 

\begin{figure}[!h]
  \centering
  
  \includegraphics[scale=0.3]{Imagens/resp_freq_4}
  \label{fig:resp_freq_4}
  \caption{Resposta em frequência do filtro passa-faixa.}
\end{figure}

A figura \ref{fig:resp_freq_db_4} mostra a resposta em dB, nota-se que a 
atenuação aumenta em aproximadamente 60 dB por década, o que corresponde a 
ordem 3 do filtro.

\begin{figure}[!h]
  \centering
  
  \includegraphics[scale=0.3]{Imagens/resp_freq_db_4}
  \label{fig:resp_freq_db_4}
  \caption{Resposta em frequência do filtro passa-faixa em dB.}
\end{figure}

A figura \ref{fig:resp_freq_phase_4} mostra a fase da resposta em frequência 
para o filtro passa altas, onde é possível observar uma defasagem de 270 graus, 
o que condiz com a teoria pois o filtro é de ordem 3.

\begin{figure}[!h]
  \centering
  
  \includegraphics[scale=0.3]{Imagens/resp_freq_phase_4}
  \label{fig:resp_freq_phase_4}
  \caption{Fase da resposta em frequência do filtro passa-faixa.}
\end{figure}
\newpage
\section{Discussão e Conclusão}
Comente os resultados obtidos, sua qualidade e confiabilidade. Tente justificar eventuais discrepâncias que forem observadas. Aponte sugestões para melhorar a qualidade dos dados etc. Coloque as condições resultantes da experiência. Você deve discernir claramente quais foram essas conclusões. Não coloque como conclusões afirmações (mesmo que corretas) que não decorrem diretamente da experiência realizada. Se possível, relacione essas conclusões com as de outras experiências. Verifique até que ponto os objetivos da experiência foram alcançados (teste de um modelo, aplicações etc.).
\newpage

\nocite{*}

\bibliography{bibliografia}{}
\bibliographystyle{IEEEtranN}


\end{document}