
\newpage

\section{Metodologia Experimental}

\subsection{FPB}

Projetar um FPB com resposta Butterworth de 3a. ordem utilizando apenas um 
indutor, terminações como no caso anterior, isto é, $R_S = 50 \Omega$ e $R_L = 
470 \Omega$ e $f_c = 5,4 \ kHz$. Implemente e caracterize a resposta em 
frequência do filtro passivo, determinando experimentalmente os parâmetros que 
caracterizam o filtro FPB:

\begin{itemize}
  \item frequência de corte;
  \item atenuação fora da faixa de passagem (dB/década);
  \item atenuação na faixa de passagem;
  \item defasagem ao longo de toda a faixa de frequências (de passagem e 
  rejeição).
\end{itemize}

\subsection{FPA}

Refaça o item anterior para um FPA com resposta Butterworth de
3a. ordem utilizando apenas um indutor, terminações $R_S = 470\Omega$ e $R_L = 
50\Omega$ e $f_c = 4,6 \ kHz$. Implemente e caracterize a resposta em 
frequência do filtro passivo, anotando os parâmetros que caracterizam o filtro 
FPA.

\subsection{FPF Cascata}
Conecte os dois filtros em série (cascata) observando as impedâncias e meça a 
resposta em frequência do conjunto. Qual a função de transferência 
correspondente? Quais a(s) nova(s) frequência(s) de corte.

\subsection{FPF}
Projete um filtro Butterworth com a função de transferência resultante da 
associação dos filtros do item anterior a partir dos valores tabelados para os
elementos LC de protótipo.

\begin{itemize}
  \item implemente novamente o filtro, agora utilizando os elementos de projeto 
  do item 4.
  \item compare a resposta em frequência (módulo) com a obtida no item 3a.
\end{itemize}