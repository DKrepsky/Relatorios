\newpage
\section{Discussão e Conclusão}
Com base nos resultados obtidos para o circuito, conclui-se que o calculo para o projeto de osciladores LC possui fundamento, pois a resposta obtida na simulação é bastante próxima da resposta calculada.

Foi possível observar que o circuito estudado possui pouca distorção harmônica, pois o conteúdo espectral das frequências harmônicas à frequência de oscilação é consideravelmente pequeno em comparação com o conteúdo da frequência central.

Um outro fato observado é que a ponteira de um osciloscópio gera uma interferência na frequência central do circuito, causando um pequeno erro na medição, sendo que quanto maior a capacitância e menor a resistência da ponta de prova, maior a divergência na frequência de oscilação.

Em relação a estabilidade, o circuito mostrou-se robusto para variações na tensão de alimentação, o que o torna útil para equipamentos alimentados por bateria, onde a tensão de alimentação varia conforme o tempo.

Sendo assim, vimos nesse laboratório conceitos fundamentais para o engenheiro eletricista, de modo a firmar os conhecimentos adquiridos durante as aulas teóricas.