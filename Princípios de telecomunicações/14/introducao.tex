\newpage
\section{Introdução}
Este trabalho tem como objetivo mostrar que os sinais analógicos podem ser transmitidos através de um canal digital, sendo convertidos em palavras código e recuperados na recepção. Isso traz uma vantagem em relação ao ruído, pois quando a distorção do sinal transmitido está dentro dos limites, não há distorção na saída do receptor, devido ao fato de a informação analógica estar codificada em formato digital. Se a quantidade de bits com erro for igual a 1, ainda sim podemos recuperar o sinal original através de técnicas de correção de erros.

Neste trabalho também é apresentado algumas técnicas para recuperação dos dados e do clock com o uso do módulo integrador e amostrador.