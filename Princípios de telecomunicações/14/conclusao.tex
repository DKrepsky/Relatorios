\newpage
\section{Discussão e Conclusão}
O experimento se mostrou muito para o entendimento de sistemas que transmitem dados analógicos através de um canal digital.
Foi possível observar vários fenômenos, como por exemplo a quebra do teorema de Nyquist, melhorando o entendimento do aluno no assunto.
Ficou claro também que os métodos de detecção e correção de erros, apesar de adicionar um bit a mais, torna a comunicação muito mais eficiente.
Observou-se também que é possível o tratamento de erros com um integrador, porém, o mesmo adiciona um bit de delay na transmissão.