\newpage
\begin{abstract}
\addcontentsline{toc}{section}{Resumo}

Neste trabalho foram analisadas algumas técnicas de sinalização digital, em especial a tecnica com não retorno ao zero (NRZ). Também foi estudado o ruído nas comunicações digitais, e a paridade, utilizada para identificar um bit com erro. Por ultimo, foi visto uma técnica para regeneração de clock (código bi-fase). Foi possível observar que um sinal analógico pode ser transmitido através de um canal digital, através da codificação do mesmo. Para dar mais robustez a transmissão, um bit de paridade pode ser utilizado de modo a detectar erros e que podemos regenerar o clock a partir da codificação bi-fase.
\end{abstract}
