\newpage
\begin{abstract}
\addcontentsline{toc}{section}{Resumo}

Neste trabalho foi realizado o estudo teórico de um oscilador de RF composto por indutor e capacitor e um transistor, como elemento amplificador, de forma a comprovar, em simulação computacional, a validade e as limitações do projeto de um oscilador LC, na configuração base comum, utilizando o modelo de pequenos sinais. Para isto, foi necessário determinar o indutor utilizado de modo a se obter uma frequência de oscilação de 4MHz. Após a simulação foi constatado que a frequência da onda de saída se encontra próxima da frequência calculada. Foi analisado também a variação da frequência de saída em função da tensão de alimentação do circuito, onde foi constatado que a topologia utilizada é robusta contra variação da tensão do sistema.
\end{abstract}
\newpage
\blankpage