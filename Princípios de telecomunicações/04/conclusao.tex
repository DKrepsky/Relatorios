\newpage
\section{Discussão e Conclusão}
Com base nos resultados obtidos para os circuitos 1 e 2, conclui-se que o calculo para o projeto de filtros ativos possui fundamento, pois a resposta obtida na simulação é bastante próxima da resposta calculada de acordo com a teoria.
Observou-se também que pode-se utilizar de vários filtros em cascata para obter uma determinada resposta em frequência. Isso ajuda a reduzir a ordem do filtro, tornando-os mais baratos e a melhorar o fator de qualidade (Q) do filtro. Um outro aspecto importante que foi observado é que uma onda quadrada, quando tem suas componentes harmônicas removidas por um filtro ativo, se torna uma senoide co período fundamental igual ao da onda original, porém, com pequenas distorções, devidas ao fato do filtro não ser ideal.

Sendo assim, vimos nesse laboratório conceitos fundamentais para o engenheiro eletricista, de modo a firmar os conhecimentos adquiridos durante as aulas teóricas.