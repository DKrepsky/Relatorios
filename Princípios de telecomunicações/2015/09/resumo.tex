\newpage
\begin{abstract}
\addcontentsline{toc}{section}{Resumo}

Neste trabalho foi realizado o estudo teórico e a simulação de redes adaptadoras de impedância, tendo como objetivo a analise da transferência de energia de uma fonte para uma carga, de modo a se obter a máxima transferência possível.
O estudo foi realizado tanto para redes do tipo $L$ (Banda estreita e banda larga) quando para redes do tipo $T$. Foram analisados o índice de mérito e a branda de passagem (para 3 dB de atenuação e pelo critério da transferência de potência) para os circuitos.
Foi observado que quanto maior a largura de banda da rede, menor seu fator $Q$ e que este problema pode ser amenizado utilizando-se várias redes do tipo $L$ em cascata.
\end{abstract}
