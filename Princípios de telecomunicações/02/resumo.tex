\newpage
\begin{abstract}
\addcontentsline{toc}{section}{Resumo}
Deve conter uma descrição do problema, a motivação e o método empregado e os resultados obtidos. O resumo deve, portanto, ser o último item a ser escrito em um trabalho, embora seja normalmente apresentado no início. O resumo deve ter uma estrutura independente do resto do trabalho, isto é, o leitor deve ser capaz, ao lê-lo, de ter uma ideia geral do trabalho, sem necessidade de consulta ao restante do trabalho.

Neste trabalho foi realizado o estudo teórico de filtros passivos compostos indutores e capacitores (filtros LC) de forma a analisar a resposta em frequência para filtros FPB, FPA, FPF e FPF em cascata. A metodologia utilizada consiste em realizar o projeto do filtro normalizado, transformar para o tipo de filtro requerido e simular o circuito no software Orcad.
Durante o laboratório foi possível observar as diferentes respostas em frequência para cada tipo de filtro estudado.Também foi visualizado a melhora no fator Q de filtros em cascata em relação a um filtro FPF comum.
\end{abstract}
\newpage
\blankpage