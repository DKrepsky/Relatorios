\newpage
\section{Discussão e Conclusão}
Neste experimento foi possível analisar o projeto de dois circuitos moduladores de AM/DSB, onde foi possível constatar que a teoria envolvida na análise da modulação AM é coerente e se aplica na prática.
Um dos fatores importantes observado foi em relação ao calculo do índice de modulação ($\gamma$) através do método do trapézio para quando $\gamma > 1$. Observou-se que, devido a cuva de amplitudes não ser linear, o valor obtido não foi o real.
Notório também é a diferença de qualidade, do sinal modulado, entre as duas topologias. O modulador a diodo, apesar de possuir fácil implementação e baixo custo, possui um fator Q (índice de mérito) baixo, o que faz com que mais energia seja perdida em frequências próximas à frequência da portadora e das bandas do sinal modulante.