\newpage
\section{Metodologia Experimental}

Este experimento foi dividido em 4 circuitos, o qual foram projetados utilizando a tabela de coeficientes normalizados provida pelo professor e, em seguida, foram simulados no programa computacional Cadence Orcad. Abaixo segue a descrição de cada circuito.

\subsection{FPB}
O primeiro filtro projetado foi um filtro passa-baixas de terceira ordem, com resposta do tipo Butterworth utilizando apenas 1 indutor. A frequência de corte dada é de 4,8 kHz, o resistor da fonte $R_s = 50\ \Omega$ e a carga $R_l = 470\ \Omega$.

Foram analisadas a frequência de corte, atenuação fora da faixa de passagem, atenuação na faixa de passagem e a defasagem ao longo de toda a faixa de frequências.

A figura \ref{fFPB} mostra o circuito utilizado.\begin{figure}[H]
\centering

\includegraphics[scale=0.8]{Imagens/fpb.jpg}
\label{fFPB}
\caption{Filtro desnormalizado passa-baixas.}
\end{figure}



\subsection{FPA}
O segundo filtro projetado foi um filtro passa-altas de terceira ordem, com resposta do tipo Butterworth utilizando apenas 1 indutor. A frequência de corte dada é de 4,2 kHz, o resistor da fonte $R_s = 470 \Omega$ e a carga $R_l = 50 \Omega$.

Foram analisadas a frequência de corte, atenuação fora da faixa de passagem, atenuação na faixa de passagem e a defasagem ao longo de toda a faixa de frequências.

A figura \ref{fFPA} mostra o circuito utilizado.\begin{figure}[H]
\centering
\includegraphics[scale=0.5]{Imagens/fpa.jpg}
\label{fFPA}
\caption{Filtro desnormalizado passa-altas.}
\end{figure}

\subsection{FPF em cascata}
No terceiro circuito o filtro FPB e o FPA desenvolvidos anteriormente foram conectados em cascata de forma a formar um filtro passa-faixa, onde foi também analisado a largura de banda passante do filtro.

Na figura \ref{fFPFC} é apresentado o circuito resultante da conexão dos filtro FPA e FPB, na topologia cascata.

\begin{figure}[H]
\centering
\includegraphics[scale=0.5]{Imagens/fpf1.jpg}
\label{fFPFC}
\caption{Filtro passa-faixa em cascata.}
\end{figure}

\subsection{FPF}
O quarto e ultimo filtro foi projetado para ter uma resposta em frequência semelhante a do filtro FPF em cascata. O objetivo foi de analisar as diferenças em se utilizar um filtro FPF e um filtro FPB em conjunto com um filtro FPA.

A figura \ref{fFPF} representa o filtro passa-faixa projetado.

\begin{figure}[H]
\centering
\includegraphics[scale=0.5]{Imagens/fpf2.jpg}
\label{fFPF}
\caption{Filtro passa-faixa.}
\end{figure}