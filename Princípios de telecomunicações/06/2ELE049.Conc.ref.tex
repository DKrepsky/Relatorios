\section{DISCUSS�ES E CONCLUS�ES}

Como foi constatado nos resultados das simula��es, o fator de m�rito do circuito do mixer � maior que o fator de m�rito do filtro.
Podemos observar tamb�m que, de acordo com a figura \ref{f_saida_up}, o up converter possui uma modula��o AM residual.
Ao analisar a resposta em frequ�ncia tanto do up como do down converter, notamos que o sinal de sa�da possui uma pequena quantidade de energia na primeira harm�nica, por�m, as demais possuem uma energia muito baixa, sendo assim, desprezadas.
Para o caso do mixer com buffer, podemos notar que h� uma n�tida distor��o no sinal do mixer. Essa distor��o gera uma diminui��o da energia da frequ�ncia FI e aumenta a energia das harm�nicas . Vale lembrar que o transistor utilizado n�o � o mesmo do roteiro, sendo assim, o ponto de opera��o do buffer est� deslocado do centro, causando uma interfer�ncia maior na sa�da do circuito.


\pagebreak

\addcontentsline{toc}{section}{REFER�NCIAS}
\nocite{*}
\bibliography{2ELE049.Ref}{}
\bibliographystyle{ieeetr}

