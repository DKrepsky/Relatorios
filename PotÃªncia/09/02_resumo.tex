\newpage
\begin{abstract}
\addcontentsline{toc}{section}{Resumo}

A tensão de saída dos conversores de potência não é a mesma que a calculada devido as perdas no sistema e as componentes parasitas dos dispositivos utilizados. Outra fonte de variação na tensão de saída é a mudança na potência da carga. Conforme a carga varia, a tensão de saída passa por um período transitório. Para uma melhor estabilidade na tensão de saída e reduzir os efeitos transitórios é necessário o uso de um controlador que ajusta o \textit{duty cycle} do sinal PWM enviado ao conversor.  Neste trabalho foi avaliado o uso de um controlador em malha aberta com proteção contra aquecimento, para um conversor do tipo Buck, com o emprego do CI 3524. Com o auxilio de um potenciômetro, a tensão de comparação do CI foi modificada, o que ocasionou na variação da tensão de saída do conversor. Foi testado também a proteção contra superaquecimento com um termistor, onde foi possível observar a atuação do CI no corte a conversão devido ao aumento de temperatura.
\end{abstract}
