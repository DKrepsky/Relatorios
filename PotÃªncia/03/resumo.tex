\newpage
\begin{abstract}
\addcontentsline{toc}{section}{Resumo}

Neste trabalho foi realizado o a análise da eficiência de um conversor buck. Para um ganho constante de 0.5 [V/V], foi verificado o rendimento do conversor para as tensões de entrada de 30V e 20V. Observou-se que a eficiência varia de acordo com a carga, atingindo seu máximo para uma potência próxima da potência projetada. Foi analisado também a forma de onda na chave do conversor buck, onde foi possível observar a existência de transientes de alta frequência devido as componentes parasitas dos componentes utilizados.
\end{abstract}