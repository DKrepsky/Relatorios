\newpage

\section{Resultados}

\subsection{Experimento 1}
Para Rz foi considerado que o diodo D1 necessita de uma corrente de 80mA para manter a tensão estável. Sendo assim, o valor de Rz (calculado com a eq. \ref{eqRz}) foi de $108.75 \Omega$, com valor comercial mais próximo de $120 \Omega$.

\begin{equation}
Rz = \tfrac{Vcc - Vz}{80mA}
\label{eqRz}
\end{equation}

Foram definidas as escalas $1k \Omega$, $10k \Omega$  e $100k \Omega$, de modo que quando o valor de Rx for o limite da escala, a saída Vo será 10V. Os valores encontrados para os resistores R1, R2 e R3 são $330 \Omega$, $3.3k \Omega$  e  $33k \Omega$, respectivamente.

A tabela \ref{tab1} mostra o valor das resistências testadas, o valor na saída Vo e a corrente mensurada com o amperimetro.

\begin{small}
\begin{table}[H]
\begin{center}
\caption{Valores de Vo e Iout medidos.}

\begin{tabular}{l|l|l}
\hline
Rx $[\Omega]$ & Vo [V]& Iout [$\mu A$]\\
\hline
560 & 5.24 & 818\\
\hline
820 &  7.84 & 958\\
\hline
2.2k & 2.07 & 649\\
\hline
4.7k & 4.46 & 777\\
\hline
15k & 1.43 & 615\\
\hline
47k & 4.52 & 780\\
\hline

\end{tabular}
\label{tab1}
\end{center}
\end{table}
\end{small}

\subsection{Pergunta 1}
\textbf{Qual amplificador você poderia usar para reduzir erros de offset e corrente de polarização?}

Poderia ser utilizado um AmpOp de melhor qualidade, como por exemplo amplificadores operacionais feitos com JFET (TL082).

\subsection{Experimento 2}
A tabela \ref{tab2} mostra o valor das resistências escolhidas, o ganho calculado para cada resistência e o ganho medido.

O ganho calculado foi obtido através da equação do ganho para amplificadores não-inversores (eq. \ref{eq2}).

\begin{equation}
G = 1 +\tfrac{Rx}{R8}
\label{eq2}
\end{equation}

O ganho medido foi obtido aplicando-se uma entrada senoidal de frequência 1KHz e amplitude 1mV.

Foram considerados para o projeto a capacidade de \textit{sink/source} do CI 4051 e a saturação do AmpOp.

\begin{small}
\begin{table}[H]
\begin{center}
\caption{Ganho calculado e medido para cada resistor.}

\begin{tabular}{l|l|l}
\hline
Rx $[\Omega]$ & G calculado [V/V] & G medido [V/V]\\
\hline
1k & 2 & 1.7\\
\hline
4.7k &  5.7 & 5.3\\
\hline
8.2k & 9.2 & 9.0\\
\hline
22k & 23 & 22.4\\
\hline
47k & 48 & 48.0\\
\hline
100k & 101 & 100.4\\
\hline
220k & 221 & 222.0\\
\hline
470k & 471 & 469.6\\
\hline

\end{tabular}
\label{tab2}
\end{center}
\end{table}
\end{small}


\subsection{Pergunta 2}
\textbf{Que modificação poderia ser feita no circuito anterior para o ganho unitário ser possível?}

Poderia ser utilizado dois amplificadores inversores em cascata.