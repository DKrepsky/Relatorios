\newpage
\begin{abstract}
\addcontentsline{toc}{section}{Resumo}

Neste trabalho foi realizado o estudo de um sensor de água capacitivo e de uma fonte de alimentação com entrada não-simétrica e saída simétrica. A fonte de alimentação tem como função converter a tensão de uma bateria em uma tensão simétrica utilizando-se de um amplificador operacional e um circuito \textit{push-pull}. Esta fonte foi utilizada de modo a alimentar um oscilador, o qual gera uma onda quadrada com período dependente do sensor capacitivo. Desta forma foi analisada a resposta do capacitor em função da quantidade de água presente em um recipiente. Foi, então, observado que, conforme a quantidade de água no recipiente aumenta, a capacitância também aumenta e, por consequência, a frequência do oscilador diminui.
Também foi observado que o sensor capacitivo possui uma resposta quase linear a variação de líquido no recipiente.
\end{abstract}
