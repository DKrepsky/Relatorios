\newpage
\section{Discussão e Conclusão}
Com base nos resultados obtidos, foi possível observar o uso de um sensor capacitivo para determinação no nível de água presente em um reservatório. 
De acordo com os dados da tabela \ref{t_oscdata} e da figura \ref{f_oscdata}, podemos concluir que o sensor apresenta uma característica aproximadamente linear de funcionamento, descrita de forma matemática pela equação \ref{e_sensor}. Vale notar que, devido a imprecisão dos instrumentos utilizados (como no caso da seringa) e no modo como o experimento foi executado, o valor do erro quadrático médio nos dados coletados é grande.
Contudo, o experimento mostra, de forma simples, a conversão de um tipo de propriedade física (nível de água) em uma grandeza elétrica (sinal elétrico periódico).