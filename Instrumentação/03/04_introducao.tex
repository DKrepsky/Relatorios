\newpage
\section{Introdução}
Para equipamentos eletrônicos em geral, uma fonte não simétrica muitas vezes é suficiente, porém, em sistemas de aquisição de dados é comum a necessidade de uma fonte de alimentação simétrica, devido ao uso de amplificadores operacionais. Caso o aparelho seja alimentado com um transformador, o problema é facilmente resolvido adicionando-se mais um enrolamento ao transformador. Contudo, para aparelhos alimentados por bateria, faz-se necessário um circuito que transforme a tensão não-simétrica em simétrica.
Um outro problema na área de instrumentação é a necessidade de determinar a quantidade de um fluido em um recipiente. Um dos sensores mais utilizados é o sensor capacitivo, o qual utiliza o fato de que a permeabilidade eletromagnética do ar ser diferente da permeabilidade do fluído sendo monitorado. Tais sensores podem operar medindo do tempo de carga do capacitor através de um resistor conhecido, ou, variando a frequência de um oscilador, onde o período da onda de saída depende da capacitância. Esses sensores possuem a vantagem de ser simples e baratos, porém necessitam de uma melhor calibração.