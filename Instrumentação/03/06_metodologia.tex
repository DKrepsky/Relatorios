\newpage
\section{Metodologia Experimental}

\subsection{Materiais}
O material utilizado foi:
\begin{itemize}
\item LM324;
\item BC556;
\item BC548;
\item 2 resistores de 470 $k\Omega$;
\item 1 resistor de 100 $k\Omega$;
\item 1 resistor de 6.8 $k\Omega$;
\item 1 resistor de 10 $k\Omega$;
\item fonte de alimentação ajustável;
\item multímetro;
\item software MATLAB.
\end{itemize}

Para execução do experimento, faz-se necessário executar os seguintes passos:

\begin{enumerate}
\item montar o circuito da figura \ref{f_fonte}, sendo R1 = R2 = 470 $k\Omega$ e Vin de 12V;
\item verificar se as tensões Vdd e Vss correspondem a 6 V e -6 V, respectivamente;
\item montar o circuito da figura \ref{f_sensor} com R1 = 100 $k\Omega$, R2 = 6.8 $k\Omega$. e R3 = 10 $k\Omega$;
\item conectar o sensor capacitivo, com 55 mL de água no recipiente do sensor, em C1;
\item medir a frequência de saída do oscilador;
\item adicionar 5 mL de água ao recipiente;
\item repetir os paços 5 e 6 para até 100 mL de água;
\item traçar a curva da capacitância x frequência do sensor;
\item encontrar os parâmetros de calibração do sensor.
\end{enumerate}
