\newpage
\section{Introdução}

Surtos de tensão são bastante comuns em circuitos eletrônicos. Esse fenômeno pode ser causado por vários fatores, tais como uma descarga elétrica na rede, por descarga eletrostática (ao tocar o circuito, por exemplo) e até pelo acionamento de máquinas pesadas próximo ao aparelho. Uma outra fonte de sobre tensão em instrumentos de medição acontece quando a saída do sensor ultrapassa o valor máximo da escala do instrumento.
Para proteger a entrada dos aparelhos, faz-se necessário o uso de circuitos que limitem a diferença de potencial entre os terminais do instrumento de medição, assim, evitando a queima dos mesmos.
Neste experimento foi analisada uma fora de proteção contra sobre tensão utilizando-se diodos do tipo zener, de silício e de germânio.

Um outro problema abordado neste trabalho é o condicionamento de sensores do tipo termo-resistivo, sendo possível o emprego de circuitos que geram uma corrente constante ou tensão constante. Durante o experimento foi analisado o condicionamento com corrente constante, o que torna a tensão de saída dependente da temperatura sendo aferida.