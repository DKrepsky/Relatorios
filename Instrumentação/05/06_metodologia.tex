\newpage
\section{Metodologia Experimental}

\subsection{Materiais}
O material utilizado foi:
\begin{itemize}
\item LM324;
\item BC556;
\item 2 diodos 1N4148;
\item 2 diodos de germânio;
\item 1 zener 5V1 1/4W;
\item 1 potenciômetro de 10 $k\Omega$;
\item 1 resistor de 4.7 $k\Omega$;
\item 1 resistor de 27 $\Omega$;
\item 1 resistor de 2 M$\Omega$;
\item 1 termistor NTC de 10 $k\Omega$;
\item fonte de alimentação ajustável;
\item multímetro;
\item software MATLAB.
\end{itemize}

Para execução do experimento, faz-se necessário executar os seguintes passos:

\begin{enumerate}
\item montar o circuito da figura \ref{f_zener}, calculando o valor de $R_z$ para $-2.0 \le V_{in} \le 7.1$;
\item variar a tensão de entrada em 0.6 V até atingir 7.1 V, anotando os valores de $V_o$;
\item montar o circuito da figura \ref{f_diodo} com os diodos 1N4148, calculando o valor de $R_1$ para $0 \le V_{in} \le 7.1$;
\item variar a tensão de entrada em 0.6 V até atingir 7.1 V, anotando os valores de $V_o$;
\item montar o circuito da figura \ref{f_diodo} com os diodos de gerânio, calculando o valor de $R_1$ para $0 \le V_{in} \le 7.1$;
\item variar a tensão de entrada em 0.6 V até atingir 7.1 V, anotando os valores de $V_o$;
\item montar o circuito da figura \ref{f_thermistor}, com $V_{cc} = 12V$, $P_1 = 1k$ e $R = 4.7 k\Omega$;
\item variar a temperatura e medir $V_o$;
\item montar gráficos para os dados obtidos.
\end{enumerate}
