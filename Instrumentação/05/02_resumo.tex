\newpage
\begin{abstract}
\addcontentsline{toc}{section}{Resumo}

Neste trabalho foi realizado, primeiramente, o estudo de circuitos de proteção contra sobre tensão utilizando diodos zener, de silício e de germânio. Foi montado um circuito empregando os diodos citados de forma a limitar a tensão de saída na condição de uma sobre tensão na entrada. Observou-se que os diodos atuam limitando a tensão de saída de acordo com a sua queda de tensão característica.
Em uma segunda etapa, foi estudado o comportamento de um sensor termo-resistivo, o qual foi condicionado a uma corrente contínua Este condicionamento fez com que a variação da temperatura produzisse uma diferença na tensão de saída do circuito. Foram coletados dados relacionando a temperatura à variação de tensão na saída do circuito. Com tais dados, foi possível obter a curva característica do sensor e seus parâmetros, para uma possível calibração.
\end{abstract}
