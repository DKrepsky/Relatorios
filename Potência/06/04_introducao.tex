\newpage
\section{Introdução}

Um conversor do tipo Boost é um conversor DC/DC onde a principal característica é que a tensão de saída é maior que a tensão de entrada. Essa topologia é utilizada por uma grande classe de conversores chaveados e contém no mínimo dois semicondutores (um diodo e um transistor), um elemento de armazenamento de energia (neste caso um indutor) e um filtro (um capacitor para reduzir o ripple na tensão de saída).

A alimentação do conversor Boost poder ser feita através de qualquer fonte DC, tais como baterias, paineis solares, geradores DC ou através da própria rede, depois de retificada e filtrada.
Esse conversor é muitas vezes chamado de \textit{step-up converter}, pois ele eleva (\textit{step-up}) a tensão de entrada. Nestes conversores, para que haja à conservação da energia, a corrente de saída é menor que a corrente de entrada, assim, um ganho em tensão representa uma redução da corrente disponível.

Como o objetivos dos conversores chaveados (\textit{Switched Mode Power Supply}) é a alta eficiência, faz-se necessário o uso de semicondutores de potência de alta frequência. Por isso, tais conversores só se tornaram amplamente utilizados a partir dos anos 50, onde o avanço na industria dos semicondutores tornou prático o emprego do conversor Boost em produtos comerciais, militares e  aeroespaciais. Hoje em dia, os conversores do tipo Boost são amplamente utilizados nos mais diversos aparelhos como celulares, televisores, carros e etc. 