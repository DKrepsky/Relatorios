\newpage
\begin{abstract}
\addcontentsline{toc}{section}{Resumo}

Neste trabalho foi realizado o a análise da eficiência de um conversor do tipo Boost. O estudo realizado com o protótipo Boost disponível no laboratório consistiu em calcular o ganho do conversor, medindo-se a potência de entrada e de saída do circuito, de acordo com a carga atrelada a saída do mesmo, mantendo o ganho. Observou-se que a eficiência do conversor é relativamente alta se comparada com outras topologias não-chaveadas. Nota-se também que a eficiência do circuito começa subindo de acordo com a carga, até atingir um ponto máximo e depois começa a cair novamente e é dependente do ganho em tensão.
\end{abstract}
