\newpage
\section{Discussão e Conclusão}
Com base nos resultados obtidos nas tabelas \ref{t_rend1} e \ref{t_rend2}, podemos concluir que a eficiência com que o circuito converte uma tensão de entrada Vi, para uma tensão de saída Vo, depende da potência sendo transferida, onde a maior eficiência encontra-se quando a razão cíclica está próxima de 50 \%. É notório também que, os conversores Boost possuem uma perda, diferente do resultado da equação (\ref{e_ganho}), onde foi calculado o ganho para um circuito ideal. Este fato se deve a perda inerente dos componentes do sistema.
Já na figura \ref{f_onda}, notamos que há um pequeno transiente na forma de onda de saída e na chave. Isso se deve ao fato de que os componentes e cabos utilizados possuem características parasitas (como a esr e lsr do capacitor de filtro), que provocam oscilações em alta frequência.