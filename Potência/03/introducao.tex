\newpage

\section{Introdução}
Os conversores cc/cc do tipo buck tem uma elevada eficiência (maior que70\%). Neste tipo de circuito, um elemento funciona como chave, o ideal é que ele opere ora em corte (quando então a corrente é quase nula), ora em saturação (quando a tensão entre os terminais é quase nula) assim ligando e desligando rapidamente, de forma a manter uma tensão de saída estabilizada, o produto V.I que corresponde à potência dissipada pelo transistor em condução permanece sempre baixo aumentando a eficiência da fonte. Evidentemente, na prática a potência no elemento série não é totalmente nula, mas através de técnicas de circuito adequada e a escolha de componentes melhores, esta potência pode ser reduzida a valores relativamente baixos em comparação com a dissipada nas fontes lineares, assim tendo uma maior eficiência, menor tamanho e maior leveza, entretanto, são complexos e mais caros, e o chaveamento da corrente pode causar problemas de ruído se não forem cuidadosamente suprimidos, também é importante destacar, entretanto, que a ondulação de saída em fontes chaveadas é muito maior em relação às fontes lineares (quase uma ordem de grandeza). Outro fator importante é que a eficiência de tais fontes varia de acordo com a potência de saída, tendo uma menor eficácia na conversão de energia para uma carga maior do que a projetada.