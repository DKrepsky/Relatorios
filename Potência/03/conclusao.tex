\newpage
\section{Discussão e Conclusão}
Com base nos resultados obtidos nas tabelas \ref{t_n30v} e \ref{t_n20v}, podemos concluir que a eficiência com que o circuito converte uma tensão de entrada Vi para uma tensão de saída Vo depende, da potência sendo transferida, onde, a maior eficiência encontra-se quando a potência transferida está um pouco a baixo da potência para qual o conversor foi projetado e que os conversores Buck possuem uma perda, diferente do resultado da equação \ref{e_ganho}, onde foi calculado o ganho para um circuito ideal.
Já na figura \ref{f_vch}, notamos que há um pequeno transiente na forma de onda. Isso se deve ao fato de que os componentes utilizados possuem características parasitas, como no caso da esr e lsr do capacitor de filtro, que provocam oscilações em alta frequência.

