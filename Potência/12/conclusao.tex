\newpage
\section{Discussão e Conclusão}
Com base nos resultados obtidos nas tabelas \ref{tab:boost2} e \ref{tab:flyback2}, podemos concluir que a eficiência com que o circuito converte uma tensão de entrada Vi, para uma tensão de saída Vo, é maior no conversor do tipo \textit{Flyback}. É notório também que, os conversores \textit{Boost} e \textit{Flyback} possuem uma perda considerável, diferente do resultado da equação (\ref{equ:ganho}), onde foi calculado o ganho para um circuito ideal. Este fato se deve a perda inerente dos componentes do sistema.
Notamos que há um pequeno transiente na forma de onda de saída e na chave. Isso se deve ao fato de que os componentes e cabos utilizados possuem características parasitas (como a esr e lsr do capacitor de filtro), que provocam oscilações em alta frequência.