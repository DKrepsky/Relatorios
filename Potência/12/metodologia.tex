\newpage
\section{Metodologia Experimental}

    \subsection{Materiais}
        O material utilizado para realização do experimento foi:

        \begin{itemize}
            \item 1 Módulo \textit{Boost};
            \item 1 Módulo \textit{Flyback};
            \item 1 CI 3525;
            \item 1 Amplificador operacional;
            \item 1 capacitor de 100uF;
            \item 1 capacitor de 220uF;
            \item 1 capacitor de 470nF;
            \item 1 capacitor de 1nF;
            \item 1 resistor de 22k$\Omega$;
            \item 2 resistores de 10k$\Omega$;
            \item 1 resistor de 2k2$\Omega$;
            \item 1 resistor de 1k$\Omega$;
            \item 1 resistor de 1M$\Omega$;
            \item 1 reostato de 50$\Omega$;
            \item 1 acoplador óptico TIL 111;
            \item 1 Gerador de função;
            \item 1 Osciloscópio de 2 canais;
            \item multímetros.
        \end{itemize}

        Para execução do experimento, faz-se necessário executar os passos abaixo, de acordo com o roteiro disponibilizado em sala de aula \cite{Roteiro}.
        
        \begin{enumerate}
          \item Montar o circuito da figura 1 (do roteiro em \cite{Roteiro}) e verificar o funcionamento em malha fechada (Vsaída máximo de 40V). O que observou na variação da tensão de saída com relação à entrada? Explique.
          
          \item Após verificar o comportamento do item 1, preencha a tabela \ref{tab:boost} utilizando  como carga um reostato de 50$\Omega$ / 1kW (utilizar a fonte de tensão em paralelo);
          
          	\begin{small}
              \begin{table}[H]
                \begin{center}
                  \caption{Conversor \textit{Boost}.}
                  \begin{tabular}{l|l|l|l|l}
                    \hline
                    Tensão de   &  Corrente de 	& Tensão de & Corrente de	& Rendimento [\%]	\\
                    entrada [V] &  entrada [A] 	& saída [V] & saída [A]  	& \\
                    \hline
                    20 		& 			& 40.0		& 0.60			& 	\\
                    \hline
                    20		& 			& 40.0		& 1.00			& 	\\
                    \hline
                    20		& 			& 40.0		& 1.50			& 	\\
                    \hline
                    20		& 			& 40.0		& 2.00			& 	\\
                    \hline
                    20		& 			& 40.0		& 2.50			& 	\\
                    \hline
                    30		& 			& 40.0		& 0.70			& 	\\
                    \hline
                    30		& 			& 40.0		& 1.00			& 	\\
                    \hline
                    30		& 			& 40.0		& 1.50			& 	\\
                    \hline
                    30		& 			& 40.0		& 2.00			& 	\\
                    \hline
                    30		& 			& 40.0		& 2.50			& 	\\
                    \hline
                  \end{tabular}
                  \label{tab:boost}
                  
                  \small Fonte: Roteiro \cite{Roteiro}.
                \end{center}
              \end{table}
            \end{small}
            
          \item Montar o circuito da figura 3 (do roteiro em \cite{Roteiro}) e verificar o funcionamento em malha fechada (Vsaída máximo de 40V). O que observou na variação da tensão de saída com relação à entrada no \textit{flyback}? Explique.
          
          \item Após montar o circuito da figura 3 (do roteiro em \cite{Roteiro}), preencha a tabela \ref{tab:flyback} utilizando como carga um reostato de de 50$\Omega$ / 1kW (utilizar a fonte de tensão em paralelo);
          
          \begin{small}
            \begin{table}[H]
              \begin{center}
                \caption{Conversor \textit{Flyback}.}
                \begin{tabular}{l|l|l|l|l}
                  \hline
                  Tensão de   &  Corrente de 	& Tensão de & Corrente de	& Rendimento [\%]	\\
                  entrada [V] &  entrada [A] 	& saída [V] & saída [A]  	& \\
                  \hline
                  20 		& 			& 35.0		& 0.60			& 	\\
                  \hline
                  20		& 			& 35.0		& 0.90			& 	\\
                  \hline
                  20		& 			& 35.0		& 1.30			& 	\\
                  \hline
                  20		& 			& 35.0		& 1.70			& 	\\
                  \hline
                  20		& 			& 35.0		& 2.00			& 	\\
                  \hline
                  30		& 			& 35.0		& 0.60			& 	\\
                  \hline
                  30		& 			& 35.0		& 1.00			& 	\\
                  \hline
                  30		& 			& 35.0		& 1.50			& 	\\
                  \hline
                  30		& 			& 35.0		& 2.00			& 	\\
                  \hline
                  30		& 			& 35.0		& 2.50			& 	\\
                  \hline
                \end{tabular}
                \label{tab:flyback}
                
                \small Fonte: Roteiro \cite{Roteiro}.
              \end{center}
            \end{table}
          \end{small}
          
          \item Explique o motivo pelo qual é necessário utilizar um opto-acoplador no controle. Por que no conversor \textit{Boost} ele não foi necessário?
          
          \item Caso ocorra variação na temperatura do circuito de controle, haverá variação na tensão do \textit{flyback}? Explique.
          
          \item Qual conversor obteve maior rendimento?
          
          \item Cite vantagens e desvantagens do CI 3525 em relação ao CI 3524.
        \end{enumerate}
 
         