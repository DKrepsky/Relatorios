\newpage
\begin{abstract}
\addcontentsline{toc}{section}{Resumo}

Nos sistemas de telecomunicações reais, sempre há a presença de ruído na recepção de um sinal. Um dos modelos utilizado para simular a presença de ruído em uma transmissão é o AWGN (\textit{aditive white Gaussian noise}), o qual representa o ruído térmico. Esse ruído perturba o sinal de forma a prejudicar a recuperação da informação, sendo que quanto mair a potência do ruído, maior a dificuldade em recuperar a informação original. Neste trabalho é analisado a eficiência na transmissão de dados utilizando as técnicas te modulação ASK, FSK e PSK M-ários, em um canal com ruído do tipo AWGN, de modo a avaliar a eficiência de cada uma delas. Foram realizados estudos para transmissões com e sem o uso da codificação Gray, que é uma técnica utilizada para melhorar a BER nos sistemas de telecomunicações. O critério de avaliação adotado para determinar a o desempenho do sistema foi a taxa de erro de bit (BER - \textit{Bit Error Rate}). Foi possível observar que conforme a quantidade de ruído aumenta, a taxa de erro de bit também aumenta e que a BER se aproxima bastante do valor calculado teoricamente.
\end{abstract}
