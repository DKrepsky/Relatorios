\newpage
\section{Introdução}

O método \textit{Orthogonal frequency-division multiplexing} (OFDM) é uma técnica de modulação digital baseada no uso de múltiplas frequências de portadora \cite{Proakis}. Um grande número de sub-portadoras ortogonais, igualmente espaçadas, são utilizadas para transportar dados de forma paralela, onde cada sub-portadora constitui um canal \cite{Lathi}. Cada canal é modulado com um esquema de modulação convencional (tal como QAM ou PSK) com uma taxa de símbolo baixa, mantendo a taxa de transferência total como em um esquema de transmissão de portadora única, para a mesma largura de banda.

A principal vantagem do OFDM em relação a transmissão com portadora única é a capacidade de lidar com condições severas de canal \cite{Stallings}. Por exemplo, a atenuação de frequências altas, interferência de banda estreita e desvanecimento seletivo em frequência.

A equalização de canal também é simplificada, pois a técnica OFDM é vista como vários sinais de baixa frequência com banda estreita em vez de um único sinal de alta frequência e banda larga. A baixa taxa de transferência utiliza um intervalo de guarda (banda de guarda) entre os símbolos, de modo a eliminar a interferência entre símbolos (ISI).

Também é possível utilizar ecos e espalhamento temporal para produzir um ganho de diversidade, melhorando a relação sinal ruído (SNR).

Algumas das aplicações para OFDM são \cite{Proakis}:

\begin{itemize}
  \item Televisão Digital;

  \item Internet DSL;
  
  \item Redes wireless;
  
  \item Redes 4G.
\end{itemize}

Assim, este trabalho explora a técnica OFDM, com modulação 16-QAM e 4-QAM.