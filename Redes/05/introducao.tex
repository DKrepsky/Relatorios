\newpage
\section{Introdução}
A modulação Delta foi introduzida nos anos de 1940 como uma forma simplificada da modulação PCM (\textit{pulse code modulation}), a qual requeria um conversor A/D de difícil implementação. A saída de um modulador Delta é uma sequência de pulsos de amostra, em uma frequência relativamente alta (em torno de 100kbit/s para um sinal de áudio com banda de 4 kHz). O valor de cada bit é determinado de acordo com a variação da amplitude do sinal da mensagem, ou seja, se o sinal de mensagem aumentou ou diminuiu, em relação com a amostra anterior. A modulação Delta é um exemplo de modulação por código de pulso diferencial (DPCM) e seu uso é mais comum em sistemas se áudio, onde a qualidade do som não é o critério principal de design.
