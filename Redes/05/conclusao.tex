\newpage
\section{Discussão e Conclusão}
As simulações tiveram resultados condizentes com a teoria, comprovando na prática, as características de operação da modulação Delta. É super interessante observar essa modulação na prática para ter convicção e compreensão de seus funcionamentos. As primeiras imagens de cada item na parte experimental provam todos esses funcionamentos.

Sobre o diagrama de olho, o mesmo se mostra uma ferramenta muito útil na análise do sinal transmitido, possibilitando verificar a qualidade do sinal obtido e até mesmo error de design no projeto.

Quanto aos receptores, foi possível observar que o receptor de correlação para sinais assimétricos em banda passante se mostrou com melhor desempenho em relação ao receptor de correlação simples, porém, o mesmo é mais complexo e, sendo assim, mais caro de se implementar na prática, sendo que a determinação de qual receptor utilizar se dará de acordo com as características do projeto em desenvolvimento.