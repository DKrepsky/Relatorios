\newpage
\section{Discussão e Conclusão}
As simulações tiveram resultados condizentes com a teoria, comprovando, na 
prática, as características de operação da modulação PAM com codificação 
duobinário. É super interessante observar essa modulação na prática para ter 
convicção e compreensão de seu funcionamento. As primeiras imagens de cada item 
na parte experimental provam todos esses funcionamentos.

De acordo com as simulações, os resultados obtidos mostraram que o sistema com 
sinalização duobinário simples possui uma menor taxa de erro de bit, quando 
comparado com o sistema com sinalização duobinário simples. Contudo, os 
requisitos de canal próximo de 0Hz são melhores no duobinário modificado.

Também foi observado o relaxamento em torno de 0Hz para o sistema duobinário 
modificado, possibilitando sua utilização em canais que não possuem uma 
resposta em frequência adequada em torno de 0Hz.