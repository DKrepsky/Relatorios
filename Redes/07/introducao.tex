\newpage
\section{Introdução}
A modulação PAM (\textit{Pulse Amplitude Modulation}) é uma das formas mais 
simples de modulação digital. Seu funcionamento consiste em inserir a 
informação codificada na amplitude do sinal portador. A demodulação é feita 
detectando-se o nível de amplitude do sinal da portadora, para cada período de 
símbolo.

Existem dois tipos de PAM: polaridade única e dupla polaridade. Na polaridade 
única é adicionado um \textit{offset} DC para garantir que todos os pulsos 
estão acima de 0V. Já polaridade dupla, os pulsos podem assumir valores 
positivos ou negativos\cite{Stallings}.
 
Apesar de simples, a modulação PAM é bastante utilizada\cite{Alencar}. Abaixo 
segue uma lista 
com as suas aplicações:

\begin{itemize}
  \item \textbf{Ethernet:} Algumas versões do protocolo Ethernet utilizam a 
  modulação PAM.

  \item \textbf{Lâmpadas de LED:} A modulação PAM é utilizada para controlar o 
  brilho de algumas lâmpadas de LED.
  
  \item \textbf{TV Digital:} É utilizada no protocolo \textit{8VSB} pelos 
  padrões do \textit{Advanced Television System Committee} para transmissão em 
  tv digital.
  
  \item \textbf{Foto biologia:} O conceito é utilizado em equipamentos que 
  fazem medidas espectrofluorométricas, sendo empregado no estudo da 
  fotossíntese em plantas.
\end{itemize}

Obviamente, a modulação PAM é utilizadas em outras situações além das listadas 
acima. Deste modo, este trabalho explorar a modulação PAM, com sistema de 
codificação duobinário.